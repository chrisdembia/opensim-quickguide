\documentclass[tikz]{standalone}
% Use the Helvetica font as the sans-serif font.
%\usepackage{helvet} % [scaled]?
\usepackage[T1]{fontenc}
%\usepackage[charter]{mathdesign}
%\def\ttdefault{blg}
\usepackage{inconsolata}
\usepackage{lmodern}
%\usepackage[scaled]{beramono}
%\usepackage[scaled]{ulgothic}
\usepackage{tikz}
\usetikzlibrary{positioning}
\usepackage{listings}
\begin{document}

\lstset{basicstyle=\ttfamily}

% Use sans-serif font by default.
% TODO \renewcommand*\familydefault{\sfdefault}

% Use 8pt sans-serif font in tikzpictures.
% TODO instead of 8pt, use scaling.
% TODO \newcommand{\fontstyle}{\fontsize{8pt}{12}\selectfont\sffamily}
% TODO \tikzstyle{every picture}+=[font=\fontstyle]

\newcommand\tikzmark[2]{
    \tikz[remember picture, overlay]\node (#1) {};
}
%!\tikzmark{A}{A}!

\lstset{language=C++, commentstyle=\color{black!30!green}\ttfamily}
\begin{tikzpicture}%[remember picture, overlay]
\node (classdeclare) [draw] {
\begin{lstlisting}[language=c++, escapechar=!]
class MyComponent : public OpenSim::ModelComponent {
  OpenSim_DECLARE_CONCRETE_OBJECT(MyComponent, OpenSim::Component);
public:
  OpenSim_DECLARE_PROPERTY(name, T, comment);
  OpenSim_DECLARE_UNNAMED_PROPERTY(T, comment);
  OpenSim_DECLARE_OPTIONAL_PROPERTY(name, T, comment);
  OpenSim_DECLARE_LIST_PROPERTY(name, T, comment);
  // See also: LIST_PROPERTY_ATLEAST, _ATMOST, and _SIZE

  void constructConnectors() {
      constructConnector<T>(name);
  }

  OpenSim_DECLARE_OUTPUT(name, T, function, dependsOnStage);
  OpenSim_DECLARE_LIST_OUTPUT(name, T, function, dependsOnStage);
  OpenSim_DECLARE_OUTPUT_FOR_STATE_VARIABLE(stateVarName);

  OpenSim_DECLARE_INPUT(name, T, requiredAtStage, comment);
  OpenSim_DECLARE_LIST_INPUT(name, T, requiredAtStage, comment);
protected:
  // The next 4 methods are invoked by initSystem() in the order shown below:
  void extendFinalizeFromProperties() override {
      Super::extendFinalizeFromProperties(); // first, invoke superclass
      // Ensure properties have appropriate values.
      if (get_default_activation() < 0) throw Exception();
  }
  void extendConnectToModel(Model& m) override {
     Super::extendConnectToModel(m);
     // Find other components in the model that you may depend on.
     // (Now replaced by Connectors.)
  }
  void extendAddToSystem(SimTK::MultibodySystem& sys) const override {
    Super::extendAddToSystem(sys);
    // Add resources (state variables, events) to the Simbody System.
    addStateVariable("activation"); addCacheVariable("fiber_force");
  }
  void extendInitStateFromProperties(SimTK::State& s) const override {
    Super::extendInitStateFromProperties(s);
    setStateVariableValue(s, "activation", get_default_activation());
  }

  // This is invoked during forward integration:
  void computeStateVariableDerivatives(const SimTK::State& s) const override {
    Super::computeStateVariableValueDerivatives(s);
    setStateVariableDerivativeValue(s, "activation", ...);
  }
  // This is invoked by special analyses, like induced accelerations:
  void extendSetPropertiesFromState(const SimTK::State& s) const override {
    Super::extendSetPropertiesFromState(s); 
    set_default_activation(getStateVariableValue(s, "activation"));
  }
};
\end{lstlisting}
};
\node (apicall) [draw, left=of classdeclare.north west, anchor=north east] {
\begin{lstlisting}[language=c++, escapechar=!]

Defines "Self", "Super", getConcreteClassName(), etc.

Properties can be either "simple" (T = double, Vec3, etc.)
or "Object" (T = Function, or any other Object).
Unnamed properties use the type T as their name.
Optional properties can have no value.
List properties contain a list of objects of type T.









What happens when you call initSystem()?

SimTK::State initSystem() {

  finalizeFromProperties();
    // clears all pre-existing connections.
    // invokes extendFinalizeFromProperties().


  connect();
    // connects all Connectors to the specified Objects,
    // and all Inputs to the specified Outputs.
    // invokes extendConnectToModel()/extendConnect().

  createMultibodySystem(); addToSystem();
    // Creates the Simbody MultibodySystem and then
    // invokes extendAddToSystem() so components can add
    // resources to the Simbody System.

  initializeState(); initStateFromProperties();
    // Create a Simbody State, and 
    // invoke extendInitStateFromProperties() to set the
    // default value of each state variable.
}
\end{lstlisting}
};
%\node [draw, left=of classdeclare.north west,
%        anchor=north east, text width=20em, align=left] (anno) {
%     SimTK::State initSystem() { TODO use tikzmark
%         finalizeFromProperties();
%         connect();
%         createMultibodySystem(); addToSystem();
%         initializeState(); initStateFromProperties();
%     }
%     hello \newline
%     x \newline
%     x  \newline 
%     x  \newline
%     x \newline
%     \newline
%     \newline
%     \newline
%     x  \newline
%};
%\draw (anno) -- (A);
\end{tikzpicture}

\end{document}


    OpenSim_DECLARE_LIST_PROPERTY_SIZE(name, T, size, comment);
    OpenSim_DECLARE_LIST_PROPERTY_ATLEAST(name, T, minSize, comment);
    OpenSim_DECLARE_LIST_PROPERTY_ATMOST(name, T, maxSize, comment);
    OpenSim_DECLARE_LIST_PROPERTY_RANGE(name, T, minSize, maxSize, comment);


