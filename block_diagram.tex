\documentclass[tikz]{standalone}
% Use the Helvetica font as the sans-serif font.
\usepackage{helvet} % [scaled]?
\usepackage[T1]{fontenc}
\usepackage{tikz}
\usetikzlibrary{graphdrawing}
\usetikzlibrary{graphs}
\usetikzlibrary{positioning}
\usetikzlibrary{matrix}
\usetikzlibrary{fit}
\usetikzlibrary{shapes.symbols}
\usetikzlibrary{shapes.misc}
% TODO \usetikzlibrary{calc}
% TODO \usetikzlibrary{matrix}
\usegdlibrary{layered}
\usegdlibrary{circular}
\usegdlibrary{trees}
\usegdlibrary{force}
\begin{document}

% Use sans-serif font by default.
\renewcommand*\familydefault{\sfdefault}

% Use 8pt sans-serif font in tikzpictures.
% TODO instead of 8pt, use scaling.
\newcommand{\fontstyle}{\fontsize{8pt}{12}\selectfont\sffamily}
\tikzstyle{every picture}+=[font=\fontstyle]

\begin{tikzpicture}[
    ]

    \node (ground) [draw] {Ground "ground"};

    \node (j1n) [below=6ex of ground] {Joint "j1"};
    \node (c1) [draw, below of=j1n, inner ysep=1em, inner xsep=2em]
    {Coordinate "c1"};
    \node (j1) [draw, fit={(j1n) (c1)}, inner sep=0.5em] {};
    \node (c1v) [draw, signal, signal to=east, fill=white] at (c1.east)
    {value};

    \node (body1) [draw, below=4ex of j1] {Body "b1"};

    \node (j2n) [below=5ex of body1] {Joint "j2"};
    \node (c2) [draw, below of=j2n, inner ysep=1em, inner xsep=2em]
    {Coordinate "c2"};
    \node (j2) [draw, fit={(j2n) (c2)}, inner sep=0.5em] {};
    \node (c2v) [draw, signal, signal to=east, fill=white] at (c2.east)
    {value};

    \node (body2) [draw, below=4ex of j2] {Body "b2"};

    % Actuator.
    \node (actu1) [draw, align=center, left=of c1, inner ysep=1em, inner xsep=2em]
    {CoordinateActuator\\"ca1"};
    \draw (actu1.east) -- (c1.west);
    \node (actu1conn)
    [draw, rounded rectangle, rounded rectangle east
    arc=concave, rounded rectangle west arc=none, fill=white] at (actu1.east)
    {coord};
    \node (actu1in)
    [draw, signal, signal from=west, signal to=nowhere, fill=white]
    at (actu1.west) {control};
    
    \node (actu2) [draw, align=center, left=of c2, inner ysep=1em, inner xsep=2em]
    {CoordinateActuator\\"ca2"};
    \draw (actu2.east) -- (c2.west);
    \node (actu2conn)
    [draw, rounded rectangle, rounded rectangle east
    arc=concave, rounded rectangle west arc=none, fill=white] at (actu2.east)
    {coord};
    \node (actu2in)
    [draw, signal, signal from=west, signal to=nowhere, fill=white]
    at (actu2.west) {control};

    \node (contr) [draw, align=center, left=22em of body1, inner ysep=2em, inner xsep=2em]
    {FeedbackController\\"fbc"};
    \node (contrin)
    [draw, signal, signal from=west, signal to=nowhere, fill=white]
    at (contr.west) {desired};
    \matrix at (contr.east) {
        \node (controut1) [draw, signal, signal to=east, fill=white]
            {u1}; \\
        \node (controut2) [draw, signal, signal to=east, fill=white]
            {u2}; \\
    };

    \draw (controut1.east) to[out=0, in=180] (actu1in.west);
    \draw (controut2.east) to[out=0, in=180] (actu2in.west);

    % Data source.
    \node (datasrc) [draw, align=center,left=5em of contr, inner ysep=1em, inner xsep=2em]
    {TableSource\\"traj"};
    \node (datasrcout) [draw, signal, signal to=east, fill=white]
    at (datasrc.east) {output};

    \draw (datasrcout.east) -- (contrin.west);

    % Reporter.
    \node (rep) [draw, right=8.5em of body1, inner sep=2em] {TableReporter};

    % Reporter inputs.
    \matrix at (rep.west) {
        % TODO make this a "multi input"
        \node (repin0) [draw, signal, signal from=west, signal to=nowhere,
            fill=white]
            {input}; \\
        \node (repin1) [draw, signal, signal from=west, signal to=nowhere,
            fill=white,
            ]
            {input}; \\
    };

    % The to[out, in] creates a curved path; the 0 and 180 are angles.
    \draw (c1v.east) to[out=0, in=180] (repin0.west);
    \draw (c2v.east) to[out=0, in=180] (repin1.west);

    \draw (ground.south) -- (j1.north);
    \draw (body1.north) -- (j1.south);

    \draw (body1.south) -- (j2.north);
    \draw (body2.north) -- (j2.south);
    
    \node (j1A) [draw, rotate=90, rounded rectangle, rounded rectangle west
    arc=concave, rounded rectangle east arc=none, fill=white] at (j1.south)
    {B};
    
    \node (j1B) [draw, rotate=-90, rounded rectangle, rounded rectangle west
    arc=concave, rounded rectangle east arc=none, fill=white] at (j1.north)
    {A};
    
    \node (j2A) [draw, rotate=90, rounded rectangle, rounded rectangle west
    arc=concave, rounded rectangle east arc=none, fill=white] at (j2.south)
    {B};
    
    \node (j2B) [draw, rotate=-90, rounded rectangle, rounded rectangle west
    arc=concave, rounded rectangle east arc=none, fill=white] at (j2.north)
    {A};


%    \node (c1) [draw, below=of ground] {Coordinate "c1"};
%    \node (j1) [draw, fit=(c1)] {Joint "j1"};
%    \node (j2) [draw, below=of j1] {Joint "j2"};

%    \node [draw, signal, signal from=west, signal to=nowhere] at (0, 0) {input};
%
%%    \node [draw, signal, signal from=south, signal to=nowhere] at (2, -2) {connector};
%
%    \node [draw, rotate=90, rounded rectangle, rounded rectangle west
%    arc=concave, rounded rectangle east arc=none] at (2, -2)
%    {connector};
%    \node [draw, rotate=90, rounded rectangle, rounded rectangle west
%    arc=concave, rounded rectangle east arc=none] at (2, -2)
%    {\rotatebox{-90}{connector}};

\end{tikzpicture}

\end{document}




%    \node (repins) at (rep.west) {
%        \tikz{
%            \node (repin0) [draw, signal, signal from=west, signal to=nowhere,
%                fill=white]
%                at (rep.west) {input};
%            \node (repin1) [draw, signal, signal from=west, signal to=nowhere,
%                fill=white,
%                below=1ex of repin0]
%                {input};
%        }
%    };

